%%%%%%%%%%%%%%%%%%%%%%%%%%%%%%%%%%%%%%%
% Deedy CV/Resume
% XeLaTeX Template
% Version 1.0 (5/5/2014)
%
% This template has been downloaded from:
% http://www.LaTeXTemplates.com
%
% Original author:
% Debarghya Das (http://www.debarghyadas.com)
% With extensive modifications by:
% Vel (vel@latextemplates.com)
%
% License:
% CC BY-NC-SA 3.0 (http://creativecommons.org/licenses/by-nc-sa/3.0/)
%
% Important notes:
% This template needs to be compiled with XeLaTeX.
%
%%%%%%%%%%%%%%%%%%%%%%%%%%%%%%%%%%%%%%

\documentclass[letterpaper]{deedy-resume} % Use US Letter paper, change to a4paper for A4 

\begin{document}

%----------------------------------------------------------------------------------------
%	TITLE SECTION
%----------------------------------------------------------------------------------------

\lastupdated % Print the Last Updated text at the top right

\namesection{Parth}{Chopra}{ % Your name
% \urlstyle{same}\url{parthc-rob.github.io} \\ % Your website, LinkedIn profile or other web address
\href{mailto:parthc@umich.edu}{parthc@umich.edu} | 361.510.0788 | \href{parthc-rob.github.io}{parthc-rob.github.io} % Your contact information
}

%----------------------------------------------------------------------------------------
%	LEFT COLUMN
%----------------------------------------------------------------------------------------

\begin{minipage}[t]{0.25\textwidth} % The left column takes up 33% of the text width of the page

%------------------------------------------------
% Education
%------------------------------------------------

\section{Education} 

\subsection{University of Michigan }

\descript{MS in Robotics}
\location{Expected Apr 2019 | Ann Arbor, MI}
% \location{ GPA: 3.23/4.0}
\sectionspace % Some whitespace after the section

\subsection{Delhi Tech. University [DTU]}

\descript{BTech in Engineering Physics}
\descript{Minor in Robotics Engineering}
\location{June 2016 | New Delhi, India}
% \location{ GPA: 3.7 / 4.0}

\sectionspace % Some whitespace after the section

%------------------------------------------------

% \subsection{La Martiniere for Boys}

% \location{Grad. May 2011 | Kolkata, India}

% \sectionspace % Some whitespace after the section

%------------------------------------------------
% Links
%------------------------------------------------

\section{Links} 

Github:// \href{https://github.com/parthc-rob}{\bf parthc-rob} \\
LinkedIn:// \href{https://www.linkedin.com/in/parth-chopra-b7414068}{\bf parth-chopra} \\
% YouTube:// \href{https://www.youtube.com/user/DeedyDash007}{\bf DeedyDash007} \\
AngelList:// \href{https://angel.co/parth-chopra-2}{\bf @parth-chopra-2} \\
% Quora:// \href{https://www.quora.com/Debarghya-Das}{\bf Debarghya-Das}

\sectionspace % Some whitespace after the section

%------------------------------------------------
% Coursework
%------------------------------------------------

\section{Coursework}

\subsection{Graduate}
Probabilistic Robotics & SLAM [current]
Advanced Artificial Intelligence \\
Self-Driving Cars: Perception \& Control \\
Computer Vision \\
Machine Learning \\
Motion Planning[audit] \\
Computational Data Science[audit] \\
Robot Systems Laboratory \\
Robot Kinematics \& Dynamics

\sectionspace % Some whitespace after the section

%------------------------------------------------

\subsection{Undergraduate}

Biophysics \\
Communication Systems \\
Digital Electronics \\
Engineering Mechanics \\
Microprocessor Interfacing \\
Numerical Methods \\
Signals \& Systems \\
\sectionspace
\subsection{Coursera/EdX}

Aerial Robotics \\
Intro to CS using Python \\
Machine Learning \\
Probabilistic Graphical Models \\
% {\footnotesize \textit{\textbf{(Research Asst. \& Teaching Asst) }}} \\
% Unix Tools and Scripting

\sectionspace % Some whitespace after the section

%------------------------------------------------
% Skills
%------------------------------------------------

\section{Skills}

\subsection{Programming}

\location{Over 5000 lines:}
Python \textbullet{} C++ \textbullet{} Matlab \textbullet{}ROS \textbullet{} \LaTeX\ \\ 
\sectionspace
\location{Over 1000 lines:}
OpenCV \textbullet{} PyTorch \textbullet{} JavaScript \textbullet{} Java \\
Point Cloud Library \\
\sectionspace
\location{Familiar:}
TensorFlow \textbullet{} Docker \textbullet{} AWS \\
NodeJS \textbullet{} Arduino \textbullet{} Solidworks


% C, C++, Python, OpenCV, MATLAB, Linux, ROS, Gazebo, PCL, PyTorch, TensorFlow, JavaScript, Java, LaTeX, Git, Docker, Ardupilot/Pixhawk, Solidworks
\sectionspace % Some whitespace after the section
\section{Awards}
\textbullet{} \location{\href{http://parthc-rob.github.io/}{Honda Mobility Hacks 2019}} \\
First Place - Project InsurCent \\
\textbullet{} \location{\href{http://www.auvsi-suas.org/competitions/2014/}{AUVSI SUAS 2014}}\\
Third Place / 33 international teams\\
Best Oral Presentation \\
\textbullet{} \location{\href{http://www.auvsi-suas.org/competitions/2013/}{AUVSI SUAS 2013}} \\ Sixth Place / 35 international teams
%----------------------------------------------------------------------------------------

\end{minipage} % The end of the left column
\hfill
%
%----------------------------------------------------------------------------------------
%	RIGHT COLUMN
%----------------------------------------------------------------------------------------
%
\begin{minipage}[t]{0.73\textwidth} % The right column takes up 66% of the text width of the page
% \fontsize{12}{10pt}\selectfont
%------------------------------------------------
% Experience
%------------------------------------------------

\section{Experience}
% Magna International
% Robotics Engineering Intern, Corporate R&D
% New Delhi, India
% Aug 2012 – Jun 2016

\runsubsection{Magna International}
\descript{| Robotics Engineering Intern, Corporate R\&D}

\location{ May 2018 – Aug 2018 | Troy, MI}
\vspace{\topsep} % Hacky fix for awkward extra vertical space
\begin{tightitemize}
\item Developed and evaluated safety-critical real-time software pipeline to control industrial manipulators for manufacturing processes primarily using ROS2.0, DDS communications, PCL and ROS-Kinetic libraries in a Linux Environment.
\item Wrote ROS packages in Python/C++ for sensor-fusion, obstacle detection and rule-set execution to interface with ABB robots in a factory setting.
\item Coordinated code development in team of 5+, performed Git repo maintenance and code documentation using Scrum and Kanban techniques.
\end{tightitemize}

\sectionspace % Some whitespace after the section

%------------------------------------------------

\runsubsection{\href{https://www.eu-robotics.net/robotics_league/news/press/open-source-arena-simulator-launched-for-erl-service-robots.html?changelang=5}{Instituto Superior T\'ecnico - Intelligent Robots and Systems Group}}
\newline
\descript{Research Intern, SocRob@Home Team}

\location{May 2017 – Jul 2017 | Lisbon, Portugal}
\begin{tightitemize}
\item Developed a ROS - Gazebo-based simulation for the IDMind MOnarCH mobile robot, worked with team of 5+ graduate students to integrate packages used for mobile navigation and task execution for RoboCup@Home competition challenges.
% \item Created a backbone.js-like framework for the Captions editor.
% \item All code was reviewed, perfected, and pushed to production.
\end{tightitemize}

\sectionspace % Some whitespace after the section

%------------------------------------------------

\runsubsection{Bubblefly Technologies}
\descript{| R\&D Engineer \& Drone Pilot}

\location{Jun 2016 – Feb 2017 | New Delhi, India}
\begin{tightitemize}
\item Developed product, concept of operations, design requirements for ground-surveying multirotor applications.
% \item I created the Meme generator, the entire Lipsum application, ported Tokens to different apps, fixed many bugs and more in PHP and Shell.
% \item Led a team from MIT, Cornell, IC London and UHelsinki for the project.
\end{tightitemize}
\vspace{-5pt}
% Bubblefly Technologies Pvt Ltd New Delhi, India R&D Engineer, Drone Pilot Jun 2016 – Feb 2017
% • Developed product, concept of operations and design requirements for applications for ground-surveying multirotors

% \sectionspace % Some whitespace after the section

%------------------------------------------------
% Research
%------------------------------------------------

\section{Research}

\runsubsection{Lab for PROGRESS}
\descript{| Graduate Research}

\location{Oct 2018 – Present | Ann Arbor, MI}
% Worked with \textbf{\href{http://www.cs.cornell.edu/~ashesh/}{Ashesh Jain}} and \textbf{\href{http://www.cs.cornell.edu/~asaxena/}{Prof Ashutosh Saxena}} to create \textbf{PlanIt}, a tool which learns from large scale user preference feedback to plan robot trajectories in human environments. Publication submitted.
\begin{tightitemize}
\item Adapted PointNet architecture to implement deep learning classifier to use Light-Field View [plenoptic] images of translucent objects to detect graspable handle-like features.
% \item Used grasp-pose in a robotic manipulation pipeline to pick objects.
\item Implemented PyTorch-based classifier for a probabilistic Depth-Likelihood Volume structure to correlate volume features for translucent objects with corresponding ground truth point-clouds for opaque objects.
\end{tightitemize}
\sectionspace % Some whitespace after the section

%------------------------------------------------

\runsubsection{Graduate Course Projects}
\newline
\descript{\href{https://github.com/parthc-rob/rob535_team9_perception}{Self-Driving Cars: Perception \& Control}}
\begin{tightitemize}
% \location{Mar 2012 – May 2013 | Ithaca, NY}
% Lead the development of \textbf{QuickTongue}, the first ever breakthrough tongue-controlled game with \textbf{\href{http://conf.ling.cornell.edu/~tilsen/}{Prof Sam Tilsen}} to aid in Linguistics research. Publication submitted.
\item Implemented algorithms for point cloud registration, visual odometry, stereo perception and SLAM
\item Implemented and trained ensemble-based Deep Learning models in PyTorch on GPU-based AWS EC2 instances to classify images, optimized performance for >99\% validation accuracy on photorealistic simulation data.
\item Implemented LQR, MPC and Quadratic Programming based methods for trajectory generation and control of simulated vehicle.
\end{tightitemize}
\sectionspace % Some whitespace after the section

\descript{Advanced Artificial Intelligence}
\begin{tightitemize}
\item Implemented a Kenken puzzle solver with various discrete search methods; Implemented Monte Carlo sampling methods for inference on Probabilistic Graphical Models.
\end{tightitemize}

\descript{\href{http://parthc-rob.github.io/}{Computer Vision: Michigan Go}}
\begin{tightitemize}
\item Adapted Amazon Go’s concept, developed vision pipeline to detect, label and track grocery-style objects and people from training data using transfer-learning with pre-trained convNet, segmentation and SIFT features.
\end{tightitemize}

\descript{Robot Systems Laboratory}
\begin{tightitemize}
\item Programmed computer vision pipeline for RGB-D sensor to execute pick-and-place tasks using 4-DOF manipulator with custom gripper.
\item Executed IMU-based motion control, odometry, A* path planning on 2-wheel segway robot using Cascaded PID control.
\item Programmed sensor model, occupancy grid map for SLAM execution and map-building on robot.
\end{tightitemize}

\descript{\href{autorob.github.io}{Robot Kinematics \& Dynamics }}
\begin{tightitemize}
\item Used a Fetch Robot simulation in JS/Three.JS to implement Forward Kinematics, Inverse Kinematics, Grid \& Sampling-Based planning algorithms (RRT-Connect, RRT*).
\end{tightitemize}


\runsubsection{{\href{http://uasdtu.com/}{Lockheed Martin-DTU - Unmanned Air Systems Student Team}}}

\descript{Avionics Division Lead, Flight Safety Officer, Avionics Engineer}
Oct 2012 – Jun 2015 | New Delhi, India
\begin{tightitemize}
\item Developed avionics for three design-development-test cycles for Unmanned Aerial System (UAS) platforms with Intelligence, Surveillance \& Reconnaissance capabilities.
\item Led multidisciplinary team of 20+ undergraduate students to compete in AUVSI Student UAS Competition hosted at NAVAIR base, Maryland – 3rd in SUAS 2014, 6th in SUAS 2013.
\end{tightitemize}
% Robot Systems Laboratory [ROB 550] Sep 2017 – Dec 2017
% • Programmed computer vision pipeline for RGB-D sensor to execute pick-and-place tasks using a 4DOF manipulator
% • Executed IMU-based motion control, odometry and A* path planning on a 2-wheel segway robot using Cascaded PID control
% • Programmed sensor model and occupancy grid map for SLAM execution on a differential-drive robot
% Robot Kinematics & Dynamics [ME 567: autorob.github.io] Sept 2017 – Dec 2017
% • Used a Fetch Robot simulation in JS/Three.JS to implement FK, IK, grid & sampling-based planning algorithms (RRT-Connect, RRT*)
% Computer Vision [EECS 504] – Course Project: Michigan Go Feb 2018 – Apr 2018
% • Adapted Amazon Go’s concept, developed vision pipeline to detect, label and track grocery-style objects and people from training data using transfer-learning with a pre-trained convNet, segmentation and SIFT features.
%------------------------------------------------
% Awards
%------------------------------------------------

% \section{Awards} 

% \begin{tabular}{rll}
% 2014	 & top 52/2500 & KPCB Engineering Fellow\\
% 2014	 & 2\textsuperscript{nd} most points & Google Code Jam, Qualification Round\\
% 2014	 & 1\textsuperscript{st}/50 & Microsoft Coding Competition, Cornell\\
% 2013	 & National & Jump Trading Challenge Finalist\\
% 2013 & 7\textsuperscript{th}/120 & CS 3410 Cache Race Bot Tournament \\
% 2012 & 2\textsuperscript{nd}/150 & CS 3110 Biannual Intra-Class Bot Tournament \\
% 2011 & National & Indian National Mathematics Olympiad (INMO) Finalist \\
% 2010 & National & Comp. Soc. of India's National Programming Contest\\
% \end{tabular}

\sectionspace % Some whitespace after the section

%------------------------------------------------
% Societies
%------------------------------------------------

% \section{Societies} 

% \begin{tabular}{rll}
% 2014 & top 12\%ile & Tau Beta Pi Engineering Honor Society\\
% 2014 & National & The Global Leadership and Education Forum (tGELF)\\
% 2012 & National & Golden Key International Honor Society\\
% 2012 & National & National Society of Collegiate Scholars\\
% \end{tabular}

% \sectionspace % Some whitespace after the section

%----------------------------------------------------------------------------------------

\end{minipage} % The end of the right column

%----------------------------------------------------------------------------------------
%	SECOND PAGE (EXAMPLE)
%----------------------------------------------------------------------------------------

%\newpage % Start a new page

%\begin{minipage}[t]{0.33\textwidth} % The left column takes up 33% of the text width of the page

%\section{Example Section}

%\end{minipage} % The end of the left column
%\hfill
%\begin{minipage}[t]{0.66\textwidth} % The right column takes up 66% of the text width of the page

%\section{Example Section 2}

%\end{minipage} % The end of the right column

%----------------------------------------------------------------------------------------

\end{document}